% Options for packages loaded elsewhere
\PassOptionsToPackage{unicode}{hyperref}
\PassOptionsToPackage{hyphens}{url}
%
\documentclass[
]{article}
\usepackage{amsmath,amssymb}
\usepackage{iftex}
\ifPDFTeX
  \usepackage[T1]{fontenc}
  \usepackage[utf8]{inputenc}
  \usepackage{textcomp} % provide euro and other symbols
\else % if luatex or xetex
  \usepackage{unicode-math} % this also loads fontspec
  \defaultfontfeatures{Scale=MatchLowercase}
  \defaultfontfeatures[\rmfamily]{Ligatures=TeX,Scale=1}
\fi
\usepackage{lmodern}
\ifPDFTeX\else
  % xetex/luatex font selection
\fi
% Use upquote if available, for straight quotes in verbatim environments
\IfFileExists{upquote.sty}{\usepackage{upquote}}{}
\IfFileExists{microtype.sty}{% use microtype if available
  \usepackage[]{microtype}
  \UseMicrotypeSet[protrusion]{basicmath} % disable protrusion for tt fonts
}{}
\makeatletter
\@ifundefined{KOMAClassName}{% if non-KOMA class
  \IfFileExists{parskip.sty}{%
    \usepackage{parskip}
  }{% else
    \setlength{\parindent}{0pt}
    \setlength{\parskip}{6pt plus 2pt minus 1pt}}
}{% if KOMA class
  \KOMAoptions{parskip=half}}
\makeatother
\usepackage{xcolor}
\usepackage[margin=1in]{geometry}
\usepackage{longtable,booktabs,array}
\usepackage{calc} % for calculating minipage widths
% Correct order of tables after \paragraph or \subparagraph
\usepackage{etoolbox}
\makeatletter
\patchcmd\longtable{\par}{\if@noskipsec\mbox{}\fi\par}{}{}
\makeatother
% Allow footnotes in longtable head/foot
\IfFileExists{footnotehyper.sty}{\usepackage{footnotehyper}}{\usepackage{footnote}}
\makesavenoteenv{longtable}
\usepackage{graphicx}
\makeatletter
\def\maxwidth{\ifdim\Gin@nat@width>\linewidth\linewidth\else\Gin@nat@width\fi}
\def\maxheight{\ifdim\Gin@nat@height>\textheight\textheight\else\Gin@nat@height\fi}
\makeatother
% Scale images if necessary, so that they will not overflow the page
% margins by default, and it is still possible to overwrite the defaults
% using explicit options in \includegraphics[width, height, ...]{}
\setkeys{Gin}{width=\maxwidth,height=\maxheight,keepaspectratio}
% Set default figure placement to htbp
\makeatletter
\def\fps@figure{htbp}
\makeatother
\setlength{\emergencystretch}{3em} % prevent overfull lines
\providecommand{\tightlist}{%
  \setlength{\itemsep}{0pt}\setlength{\parskip}{0pt}}
\setcounter{secnumdepth}{-\maxdimen} % remove section numbering
\ifLuaTeX
  \usepackage{selnolig}  % disable illegal ligatures
\fi
\usepackage{bookmark}
\IfFileExists{xurl.sty}{\usepackage{xurl}}{} % add URL line breaks if available
\urlstyle{same}
\hypersetup{
  pdftitle={Universidade de Brasília ~IE - Departamento de Estatística ~Análise de Dados Categorizados - 1/2025},
  pdfauthor={Bom Estudo!!!},
  hidelinks,
  pdfcreator={LaTeX via pandoc}}

\title{Universidade de Brasília ~IE - Departamento de Estatística
~Análise de Dados Categorizados - 1/2025}
\author{Bom Estudo!!!}
\date{29/04/2025}

\begin{document}
\maketitle

{
\setcounter{tocdepth}{2}
\tableofcontents
}
\section{Exercício - Consumo de Álcool e Mal Formação
Congênita}\label{exercuxedcio---consumo-de-uxe1lcool-e-mal-formauxe7uxe3o-conguxeanita}

A tabela a seguir apresenta os resultados de um estudo prospectivo sobre
consumo de álcool pelas mães e mal formação congênita de seus filhos.
Após três meses de gravidez, mulheres na amostra completaram um
questionário sobre consumo de álcool, medido em número médio de drinques
por dia. Após o nascimento das crianças, foram feitas observações sobre
a presença ou ausência de mal formação congênita.

\begin{longtable}[]{@{}lrrr@{}}
\caption{Consumo de Álcool e Mal formação congênita}\tabularnewline
\toprule\noalign{}
Consumo & Ausente & Presente & Total \\
\midrule\noalign{}
\endfirsthead
\toprule\noalign{}
Consumo & Ausente & Presente & Total \\
\midrule\noalign{}
\endhead
\bottomrule\noalign{}
\endlastfoot
0 & 17066 & 48 & 17114 \\
\textless1 & 14464 & 38 & 14502 \\
1-2 & 788 & 5 & 793 \\
3-5 & 126 & 1 & 127 \\
\textgreater=6 & 37 & 1 & 38 \\
\end{longtable}

\subsection{Questões}\label{questuxf5es}

\begin{enumerate}
\def\labelenumi{\arabic{enumi}.}
\item
  \textbf{Identifique as variáveis em estudo e classifique quanto ao
  tipo.}
\item
  \textbf{Identifique a variável resposta e a explicativa.}
\item
  \textbf{Determine a proporção de presença de mal formação congênita
  para cada nível de consumo de álcool e analise os resultados obtidos.}
\item
  \textbf{Verifique se a presença de mal formação congênita nos bebês
  está associada ao nível de consumo de álcool das mães a um nível de
  significância de 5\% e tratando as variáveis como qualitativas
  nominais.}

  \begin{enumerate}
  \def\labelenumii{\alph{enumii})}
  \tightlist
  \item
    Comente a decisão tomada considerando o nível de significância
    solicitado. A decisão seria a mesma para outro nível de
    significância? Qual seria sua recomendação?\\
  \item
    Os pressupostos do teste foram atendidos? O que poderia ser feito?\\
  \end{enumerate}
\item
  \textbf{Refaça o teste utilizado no item 4 agregando categorias para
  contornar o problema indicado no item 4b.}\\
  Comente a decisão tomada com relação aos aspectos considerados nos
  itens 4a e 4b.\\
\item
  \textbf{Os resultados dos testes realizados permitem concluir sobre a
  existência de tendências na associação entre as variáveis considerando
  o nível de consumo de álcool? Justifique sua resposta.}
\item
  \textbf{Construa tabelas 2 × 2 que permitam medir a associação entre
  presença de mal formação congênita para cada nível de consumo de
  álcool em relação a ausência de consumo de álcool. Comente os
  resultados. Eles sugerem alguma tendência?}
\end{enumerate}

\end{document}
