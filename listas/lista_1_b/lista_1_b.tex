% Options for packages loaded elsewhere
\PassOptionsToPackage{unicode}{hyperref}
\PassOptionsToPackage{hyphens}{url}
%
\documentclass[
]{article}
\usepackage{amsmath,amssymb}
\usepackage{iftex}
\ifPDFTeX
  \usepackage[T1]{fontenc}
  \usepackage[utf8]{inputenc}
  \usepackage{textcomp} % provide euro and other symbols
\else % if luatex or xetex
  \usepackage{unicode-math} % this also loads fontspec
  \defaultfontfeatures{Scale=MatchLowercase}
  \defaultfontfeatures[\rmfamily]{Ligatures=TeX,Scale=1}
\fi
\usepackage{lmodern}
\ifPDFTeX\else
  % xetex/luatex font selection
\fi
% Use upquote if available, for straight quotes in verbatim environments
\IfFileExists{upquote.sty}{\usepackage{upquote}}{}
\IfFileExists{microtype.sty}{% use microtype if available
  \usepackage[]{microtype}
  \UseMicrotypeSet[protrusion]{basicmath} % disable protrusion for tt fonts
}{}
\makeatletter
\@ifundefined{KOMAClassName}{% if non-KOMA class
  \IfFileExists{parskip.sty}{%
    \usepackage{parskip}
  }{% else
    \setlength{\parindent}{0pt}
    \setlength{\parskip}{6pt plus 2pt minus 1pt}}
}{% if KOMA class
  \KOMAoptions{parskip=half}}
\makeatother
\usepackage{xcolor}
\usepackage[margin=1in]{geometry}
\usepackage{graphicx}
\makeatletter
\def\maxwidth{\ifdim\Gin@nat@width>\linewidth\linewidth\else\Gin@nat@width\fi}
\def\maxheight{\ifdim\Gin@nat@height>\textheight\textheight\else\Gin@nat@height\fi}
\makeatother
% Scale images if necessary, so that they will not overflow the page
% margins by default, and it is still possible to overwrite the defaults
% using explicit options in \includegraphics[width, height, ...]{}
\setkeys{Gin}{width=\maxwidth,height=\maxheight,keepaspectratio}
% Set default figure placement to htbp
\makeatletter
\def\fps@figure{htbp}
\makeatother
\setlength{\emergencystretch}{3em} % prevent overfull lines
\providecommand{\tightlist}{%
  \setlength{\itemsep}{0pt}\setlength{\parskip}{0pt}}
\setcounter{secnumdepth}{5}
\ifLuaTeX
  \usepackage{selnolig}  % disable illegal ligatures
\fi
\usepackage{bookmark}
\IfFileExists{xurl.sty}{\usepackage{xurl}}{} % add URL line breaks if available
\urlstyle{same}
\hypersetup{
  pdftitle={Lista 1 B - Análise de Dados Categorizados - 1/2025},
  pdfauthor={Davi Wentrick Feijó},
  hidelinks,
  pdfcreator={LaTeX via pandoc}}

\title{Lista 1 B - Análise de Dados Categorizados - 1/2025}
\author{Davi Wentrick Feijó}
\date{15 de maio de 2025}

\begin{document}
\maketitle

{
\setcounter{tocdepth}{2}
\tableofcontents
}
\section{Lista de Exercícios 1b}\label{lista-de-exercuxedcios-1b}

Estude os tópicos 2.5 e 2.6 do capítulo 2 do livro-texto e as notas de
aula e resolva os seguintes exercícios:

\subsection{Exercício 2.27}\label{exercuxedcio-2.27}

A study on educational aspirations of high school students (S. Crysdale,
Int. J. Comp. Sociol., 16: 19--36, 1975) measured aspirations using the
scale (some high school, high school graduate, some college, college
graduate). For students whose family income was low, the counts in these
categories were (9, 44, 13, 10); when family income was middle, the
counts were (11, 52, 23, 22); when family income was high, the counts
were (9, 41, 12, 27).

\begin{enumerate}
\def\labelenumi{\arabic{enumi}.}
\tightlist
\item
  \textbf{Teste de independência} de aspirações e renda familiar usando
  \(X^{2}\) ou \(G^{2}\). Interprete e explique a deficiência deste
  teste para esses dados.
\item
  \textbf{Resíduos padronizados}. Eles sugerem algum padrão de
  associação?
\item
  \textbf{Teste mais poderoso}. Conduza um teste mais sensível e
  interprete os resultados.
\end{enumerate}

\subsection{Exercício 2.29}\label{exercuxedcio-2.29}

Um estudo (B. Kristensen et al., J. Intern. Med., 232: 237--245, 1992)
considerou o efeito da prednisolona na hipercalcemia severa em mulheres
com câncer de mama metastático. De 30 pacientes, 15 foram selecionadas
randomicamente para receber prednisolona e as outras 15 formaram o grupo
controle. A normalização no nível de cálcio ionizado sérico foi
alcançada por sete das 15 pacientes tratadas com prednisolona e por 0
das 15 pacientes do grupo controle.

\begin{itemize}
\tightlist
\item
  Use o \textbf{teste exato de Fisher} para encontrar um valor de \(P\)
  para testar se os resultados foram significativamente melhores para o
  tratamento do que para o controle. Interprete.
\end{itemize}

\subsection{Exercício 2.30}\label{exercuxedcio-2.30}

A Tabela 2.17 contém resultados de um estudo comparando a terapia de
radiação com cirurgia no tratamento do câncer de laringe. Use o
\textbf{teste exato de Fisher} para testar \(H_{0}: \theta = 1\) contra
\(H_{a}: \theta > 1\). Interprete os resultados.

\begin{table}[ht]
  \centering
  \caption{Dados para o Problema 2.30}
  \begin{tabular}{lcc}
    \hline
     & Cancer Controlado & Cancer Não Controlado \\
    \hline
    Cirurgia           & 21                & 2                    \\
    Terapia de Radiação & 15               & 3                    \\
    \hline
  \end{tabular}
\end{table}

Fonte: W. Mendenhall et al., Int. J. Radiat. Oncol. Biol. Phys., 10:
357--363, 1984. Reprinted with permission from Elsevier Science Ltd.

\subsection{Exercício 2.31}\label{exercuxedcio-2.31}

\begin{enumerate}
\def\labelenumi{\arabic{enumi}.}
\tightlist
\item
  Obtenha e interprete um valor de \(P\) exato de \textbf{duas caudas}.
\item
  Obtenha e interprete o valor de \(P\) \textbf{mid} unilateral. Discuta
  as vantagens deste tipo de \(P\) em comparação com o ordinário.
\end{enumerate}

\begin{center}\rule{0.5\linewidth}{0.5pt}\end{center}

\textbf{Bom Estudo!!!!}

\end{document}
